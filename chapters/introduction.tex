\chapter{Introduction}
\label{cha:Introduction}

% \textbf{This introduction will be reworked once the whole bachelor's thesis is finished, please ignore the current introduction.}

The first chapter of the bachelor's thesis will provide a summary of what this thesis is and what the reason was to write this scientific paper. After reading this chapter the reader should have a clear overview of the content of this bachelor's thesis. It should also be clear what to expect from this paper and what knowledge can be obtained by reading this thesis. The reader should have a basic understanding of general software development in order to being able to understand all explanations and discussions that are provided in this thesis. 

\section{Motivation and Reason}

Nowadays web development has gotten very important because the web is a steadily growing platform which provides multiple types of content and can be used on almost any operating system. Oftentimes it is easier to develop a web application instead of platform specific software in order to reach more users that can actually use the software. Programmers can get started with web development very fast and very easily as in most cases some kind of scripting languages are used to realize web applications. These scripting languages are not very difficult to learn. This is the very reason why there is a large amount of web developers who do not have any experience in software development nor any knowledge about software architecture that develop all kinds of web applications. Due to this fact countless software projects are being realized that are hardly maintainable or that have to be refactored if not redeveloped from scratch at a later point in time. It can cost companies lots of money and time having to invest into refactoring and redeveloping tasks. The motivation of this paper is that the reader can learn something about web development and to raise attention that application architecture is exceptionally important especially in web application projects.

\section{Goals and Overview}

The thesis is about web application development with the framework named \enquote{ReactJS}. These days the scripting language JavaScript really has taken over software development as one of the most popular scripting languages to implement all kinds of applications. The language is not only used for web development but it is also heavily utilized to program all other kinds of software like back end server applications for instance. It can also be used to realize applications for all kinds of platforms like Android or iOS for example. The aim of this thesis is to provide a general overview of what it takes to develop web applications using the whole JavaScript ecosystem in conjunction with Facebook's JavaScript front end framework ReactJS.

The paper however will also focus on a very important aspect of software development being application architecture. It is clear that application architecture is a very if not the most important part of realizing any kind of software project. The term \enquote{any kind of software} of course also includes web applications. To emphasize the importance of good application architecture even in web development, one of the goals of this thesis is to demonstrate how advantageous it is to implement a web application with a well thought through architecture. 

In 2014 Facebook introduced a supposedly new application architecture named Flux which, according to them, solves all problems that current web applications with traditional architectures like MVC suffer from. The goal of the second chapter is to neutrally compare the MVC pattern to the Flux pattern which will enable the reader of this paper to choose which technology to use to develop a web application. The major research aspect is the usability and the practicality of the Flux architecture in conjunction with a JavaScript framework, ReactJS in this case, alongside with a library called Redux that implements the Flux pattern and handles the state of the application. The paper will explain why choosing a Flux-Redux solution is a good idea to handle the state of a web application instead of implementing a traditional MVC solution.

The third chapter will introduce the Reader to the ReactJS framework. It contains every information that is necessary to improve the understanding of what the framework is about and how to get started more quickly. The chapter will also cover a library named Redux which is an implementation of the Flux pattern. Having a closer look at ReactJS in conjunction with Redux can demonstrate, how application state can be handled in client side web applications.

Last but not least, the thesis will quickly introduce the reader to the prototype that was developed as a part of this work. The goal of the prototype was to develop a big-data visualizing web application that is implemented by using the latest cutting edge web technologies. Particular emphasis was placed on producing a code base that is easily maintainable and easily understandable by developers that have to later maintain the project. By explaining the prototype, it is easier to show how best practices that will be described in this thesis are used in real world projects.

% This bachelor thesis is about the front end framework ReactJS and why everybody is using JavaScript instead of just coding bare HTML and a little bit of JQuery. The aim is to provide a general overview of developing web applications using the JavaScript eco system. This topic includes the NodeJS Node Package Manager (npm), which is probably the biggest and most wide spread package manager for JavaScript libraries and modules. First there will be a short overview of common barebone JavaScript and also the ECMAscript standard, its implementations and how it is used to create large scale web applications like Facebook or Airbnb. In the prototype and code examples features of the latest concepts of ECMAscript will be used. These features lack an official implementation, but they are widely used to transpile code to older versions of the ECMAscript standard to make the code browser compatible. This approach might seem a bit odd for a common software engineer, but isn't it right that a script language should only be interpreted and not be compiled in any way? Speaking of interpreting script code, it is also necessary to give an introduction to Webpack. Webpack is a module bundler and it is used in the prototype to handle all code’s dependencies that come from npm. It bundles all included libraries, frameworks and the prototype’s code into one big application. It is essential especially for ReactJS to use a code bundling mechanism which not only handles all dependencies, but also splits the code to several code parts to minimize the time the user has to wait until the webpage has fully loaded. 

% The focus however should be on \enquote{Web application development with ReactJS}. By explaining the eco system, its package manager, and all quirks with JavaScript, the paper finally is enabled to cover ReactJS itself and a very interesting library named Redux. One could say that ReactJS is the V in MVC and Redux is an implementation of Flux which is a coexisting software pattern to MVC that redefines the M and C aspect as well as data flow and data binding. Having a closer look at ReactJS in conjunction with Redux can demonstrate, how easy the handling of application state can be in big applications. That is the reason why the reader of this paper will dive deeper into the topic of handling application state, whether using the MVC pattern or using the Flux pattern. The comparison should be a core part of this research paper. 

% Therefore, this bachelor thesis should not only cover a front end Framework, but also neutrally compare the MVC pattern to the Flux pattern which will enable the reader of this paper to choose which technology to use to develop a web application. The major research aspect is the usability and the practicality of using a JavaScript framework, ReactJS in this case, alongside with a library that handles the state of the application. The paper will also explain, why I chose to use the Redux library to handle the state of my application instead of a MVC solution. This is one of the good aspects of ReactJS -- it allows the programmer to choose all other programming aspects and architectural approaches. The goal of this paper is to give an introduction to ReactJS and Redux. After reading this bachelor thesis, one will know all about the library and the aspect of programming front ends with it. The focus will be on the advantages of ReactJS in combination with Redux and whether there are any performance enhancements in comparison to other front end programming techniques and the use of other programming patterns like MVC. This thesis will provide all necessary information for educating the reader and convincing them to start programming user interfaces with ReactJS and Redux while remaining neutral in comparing this stack to other competing stacks. The framework itself will not be fully explained, but a comparison of its programming possibilities will be made so the actual programmers can decide themselves what to use for their own project.

% \section{Goals and Overview}

% To begin with, it is important to provide a general overview of the whole NodeJS eco system and an explanation of the popularity of its use nowadays. Furthermore, the paper will give a short introduction to JavaScript and the ECMAscript standards which are used in the prototype project. It is necessary surmise Webpack as well because it is essential in transforming the prototype from single code snippets to be interpretable by the browser. The goal is, to help the reader understand why so many people use JavaScript to develop giant web applications and why a script language that was intended to accomplish easy tasks is now used for creating applications, frameworks and libraries. After accomplishing this, it is then time to introduce the reader to ReactJS and all of its features. Once the reader has a better understanding of ReactJS, the next step is to get to explain the biggest and  most formidable part of this research paper which is the comparison of an old but well established pattern (MVC) and a new pattern (Flux) which came to be in 2014. It is also important to introduce an implementation of the Flux architecture which is called Redux. A very important library that one should use with Redux called ImmutableJS. ImmutableJS can help the modern JavaScript programmer to handle immutability which the script language itself can’t handle. Immutability is very formidable, considering the fact, that the application should only update, if the application state really has changed. It is important to represent how well these 3 libraries play together and how powerful the combination of the three libraries are. The best way to accomplish this goal is to introduce the reader to the prototype, which includes all JavaScript technologies that are described in the research paper.
