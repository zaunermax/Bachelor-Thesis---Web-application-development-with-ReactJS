\chapter{Conclusion}
\label{cha:conclusion}

This chapter is a recapitulation of all topics and discussions that were presented in this thesis. It is a conclusion of all findings that were discovered during the development of the prototype and the research phase of this paper.

\section{Results}

The thesis yields very interesting results that can be taken into consideration when having to decide which technology to utilize to start a web application software project. The comparison between Flux and MVC shows that flux is rather similar to MVC but also demonstrates that Flux is still a viable web application architecture. The thesis also gives an overview to web application development with ReactJS in general. It shows what effort it takes to start developing a web application with ReactJS and how the framework can be used to implement real world web applications. Finally, the paper concludes all findings by presenting a prototype that tries to incorporate all results and best practices that are presented in this thesis.

\subsection{Flux and MVC}

In conclusion Chapter \ref{cha:fluxreduxmvc} shows that Flux and MVC are very similar. Facebook introduced the Flux pattern as a brand new programming pattern but it turns out to be a copy of a slightly changed MVC pattern. Though it must be emphasized that Flux is indeed different to MVC as data is injected actively into the components, whereas MVC depends on data bindings to react to updated data.

There are many versions and mutations of the MVC pattern. In this thesis Flux is compared to the traditional version of MVC as it was developed in 1970 by the lisp community. It must be noted that there are many subversions of MVC that are even more similar to the Flux application architecture. The research of every possible MVC sub-pattern would have gone beyond the scope of this paper.

Even though Flux is very similar to MVC the pattern is still highly usable for developing web applications. The new pattern seems to be specially tailored to be used in conjunction with the ReactJS framework which indeed works very well. Using Flux and ReactJS results in well structured web applications that can be debugged and scaled rather easily as the thesis shows.

\subsection{Web application development with ReactJS}

As discussed in the Section \ref{ssec:jsx}, ReactJS is a very promising framework as it puts an abstraction layer to the view and lets programmers build user interfaces without having to know how the user interface elements are actually created in the background. This feature enables the framework to be used across all platforms. By mastering the declarative syntax of ReactJS the programmer is enabled to reuse components across all platforms including native mobile platforms.

The thesis also shows that ReactJS can be used to develop highly performance optimized web applications. By using all provided best practices it should ease the effort of the reader to develop a well structured, well scaling web application with ReactJS.

There thesis puts a strong emphasis on the importance of having a well thought-through application architecture. In fact, application architecture is important for all kinds of software projects but the thesis should have shown that it is also a very important aspect of web applications. As stated many times throughout this thesis, the application architecture aspect oftentimes gets omitted in web projects which is a big problem the web community currently has to deal with. This thesis tries to alleviate the problem by showing the reader that application architecture is an utterly important aspect of web development. 

\subsection{Prototype}

The Chapter \ref{cha:prototypejulepsearch} shows how ReactJS can be used in a real world scenario. The basic architecture is explained to provide a better overview to the prototype. Its motivation and the goals show that ReactJS can be used for all kinds of application scenarios. As stated in the Section \ref{sec:whyreactjs} and \ref{ssec:usedtechnologies} the prototype also shows that Facebook's framework does not make many assumptions about the development stack and can be used in conjunction with many different web technologies.

As time progresses the whole ReactJS ecosystem is evolving. Since the time of developing the prototype certain best practices were omitted and others were added. In addition to a constantly changing ecosystem the lack of experience of how to develop a well structured ReactJS application complicated the development of the prototype. The last Section \ref{ssec:donebetter} shows what could have been done better.



