\chapter*{Abstract}

The goal of this thesis is to provide an overview to web application development with JavaScript by using Facebook's front end framework ReactJS. It is directed at developers that would like to start developing web application software in general or at intermediate web developers who want to get started with ReactJS. 

%The reader should have a basic understanding of general software development in order to being able to understand all explanations and discussions that are provided in this thesis.

However, the emphasis is on raising attention that application architecture and a well structured code base is one of the most important aspects of any web application software project. Currently many software projects exist in the web that are hardly maintainable and have a bad or no application architecture at all which obviously leads to problems when the project has to be maintained over a longer time period.

The research section of this thesis is about the comparison between the application architecture paradigm called Flux which was invented by Facebook and the traditional Model View Controller (MVC) pattern. After research it is shown that the Flux pattern is very similar to the MVC pattern but still viable to be used in web application software projects.

On top of a broad introduction to the general JavaSript ecosystem and all of its tooling that is necessary to get started, this thesis also gives a specific introduction to the ReactJS library itself. It thus comprises a concise summary of all requirements and best practices of developing a web application with ReactJS.

As part of this work a prototype for a web application with ReactJS was developed and is introduced to the user. The research that is presented in this thesis is further corroborated by studying a real world example.


%Nowadays the modern web developer uses JavaScript front end libraries like ReactJS, Angular2 or VueJS for creating web applications. Oftentimes it is very difficult to decide what framework to use. Even though the same goal is achieved with all libraries, the frameworks use different software paradigms and use different approaches to solve software architectural problems. This fact makes some libraries fit better in certain situations than  others. The key is to get a rough understanding of all the different libraries and their approaches to solve problems to be able to decide what framework fits best in what problem statement.

%\today 

%This paper not only provides an overview to the JavaScript front end framework ReactJS but also introduces the reader to a new software paradigm called Flux

%as time progresses all get more similarasdfa

