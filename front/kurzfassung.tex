\chapter*{Kurzfassung}

Diese Arbeit gibt einen Überblick über die Enwicklung von Webanwendungen mit dem JavaScript Framework ReactJS, welches von Facebook erfunden wurde. Die Arbeit richtet sich einerseits an Entwickler und Programmierer, die noch wenig Erfahrung im Bereich der Web-Entwicklung haben und sich einen Überblick über die Implementierung von Single-Page-Webanwendungen verschaffen möchten. Sie richtet sich aber ebenso an fortgeschrittene Web-Entwickler, die sich für verschiedene Web-Anwendungsarchitekturen interessieren und das ReactJS Framework kennenlernen wollen.

Die Arbeit soll den Leser dafür sensibilisieren, dass eine gute Architektur einer der wichtigsten Aspekte eines Softwareprojekts im Web-Umfeld ist. Webanwendungen leiden oft unter schlechter Wartbarkeit, da es vielen Entwicklern an der nötigen Erfahrung fehlt, eine ordentliche Architektur in einem Web-Projekt zu realisieren. Diese Arbeit zeigt, dass man von guter Anwendungsarchitektur auch auf lange Hinsicht nur profitieren kann.

Der wissenschaftliche Teil dieser Arbeit beschäftigt sich mit dem Vergleich zweier Softwaremuster. Das von Facebook entwickelte Paradigma Flux wird mit dem traditionellen Model-View-Controller (MVC) Modell verglichen, welches seit den 70er-Jahren existiert. Es wird gezeigt, dass Flux viele Aspekte von MVC kopiert und die zwei Muster einander relativ ähnlich sind. Dennoch kann Flux eine sinnvolle Alternative zu herkömmlichen Web-Architekturen sein, besonders dann, wenn Flux für die Programmierung von ReactJS-Anwendungen verwendet wird.

Zusätzlich zu einer ausführlichen Einführung in das ReactJS-Framework bietet diese Arbeit auch einen Überblick über alle für die Programmierung einer fortgeschrittenen Webanwendung mit ReactJS erforderlichen Werkzeuge.

Im Rahmen dieser Arbeit wurde ein Prototyp für eine Webanwendung mir ReactJS entwickelt. Anhand dieser Beispielanwendung wird demonstriert, wie die behandelten Konzepte und Methodiken angewendet werden können.

%über sämtliche Werkzeuge die notwendig sind, um eine fortgeschrittene Web Anwendung mit ReactJS zu programmieren.

%Zu guter Letzt wird noch der Prototyp vorgestellt, der simultan zu dieser Arbeit entstanden ist. An dem Beispiel einer realen ReactJS Web Anwendung lassen sich die erläuterten Erklärungen und bewährten Methodiken besser erklären.